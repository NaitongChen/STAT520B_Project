% !TEX root = ../main.tex

% Body section

\section{Extending to Multiple Variational Families} \label{sec:extension}
In this section, we introduce our proposed extension in detail. All of the work so far has focused on increasing the expressibility of a single variational family through mixture approximations. However, the variational family can be further diversified by considering multiple distribution families when fitting the components of the mixture approximation. The key observation is from \cref{eq:component.wise}. Specifically, all of the computations on the gradient can be done component-wise. Therefore, it is possible to use different distribution families for different components. In addition, we know from \cite{kingma2013auto} that the reparameterization trick is not limited to the Gaussian distributions. As an example, all distributions with a closed-form inverse CDF function can be reparameterized through this function using samples from the uniform distribution.\\\\
One thing to note, though, is that in the $\mathbb{E}_{q_c} \left[ \ln q^{(C)}(x;\psi,\lambda) - \ln\pi(x) \right]$ term from \cref{eq:component.wise}, $q^{(C)}(x;\psi,\lambda)$ is the overall mixture approximation. If we were to consider multiple distribution families, this mixture approximation may contain components from multiple distribution families. As a result, the samples from the reparameterized distribution for component $c$ needs to be able to transform to all of the distribution families in the mixture. Therefore, one requirement for extending VB to multiple variational families is to ensure that the reparameterized reference distributions have the same support. Once this is satisfied, at each iteration, we can fit a component from each of the distribution families considered, and add the one that makes the resulting mixture yield the smallest divergence to the target. Then ideally, the characteristisc of the different distribution families can be exploited to achieve the best possible fit. As a result, we may be able to obtain a good approximation of the targe using fewer components. Note that although this extension introduces more computation before adding each component, the multiple component optimization problems can be easily parallelized to remain computationally competitive against the standard variational boosting algorithm.
% ...